\documentclass[
	a4paper,
	pagesize,
	pdftex,
	left=3cm,
	right=3cm,
	12pt,
	%twoside, % + BCOR darunter: für doppelseitigen Druck aktivieren, sonst beide deaktivieren
	%BCOR=3mm, % Dicke der Bindung berücksichtigen (Copyshop fragen, wie viel das ist)
	ngerman,
	fleqn,
	final,
	]{scrartcl}
\usepackage{ucs}
\usepackage[utf8x]{inputenc} % Eingabekodierung: UTF-8
\usepackage{fixltx2e} % Schickere Ausgabe
\usepackage[T1]{fontenc} % ordentliche Trennung
\usepackage[ngerman]{babel}
\usepackage{lmodern} % ordentliche Schriften
%\usepackage{times}
%\renewcommand{\familydefault}{\sfdefault}
%\usepackage{mtpro2} % ordentliche Schriften
%\usepackage[10]{fontspec}
%\setsansfont{Liberation Sans}
%\setmonofont[SmallCapsFont={Liberation Mono}]{Liberation Mono}
\usepackage[unicode=true]{hyperref}
\usepackage{setspace,graphicx,tikz,tabularx} % für Elemente der Titelseite
\usepackage[draft=false,babel,tracking=true,kerning=true,spacing=true]{microtype} % optischer Randausgleich etc.
\usepackage{setspace}

\usepackage{lipsum}
\usepackage[bottom,hang]{footmisc}
\setlength{\footnotemargin}{5mm}


\begin{document}

% Beispielhafte Nutzung der Vorlage für die Titelseite (bitte anpassen):
\input{Institutsvorlage}
\titel{\normalfont \textbf{XXX}\\ XXX} % Titel der Arbeit
\typ{\normalfont Hausarbeit} % Typ der Arbeit:  Diplomarbeit, Masterarbeit, Bachelorarbeit
\grad{Master of Science (M. Sc.)} % erreichter Akademischer Grad
% z.B.: Master of Science (M. Sc.), Master of Education (M. Ed.), Bachelor of Science (B. Sc.), Bachelor of Arts (B. A.), Diplominformatikerin
\autor{\normalfont Marie Musterfrau} % Autor der Arbeit, mit Vor- und Nachname
\gebdatum{\normalfont 1.1.1970} % Geburtsdatum des Autors
\gebort{\normalfont Lüneburg} % Geburtsort des Autors
%\gutachter{\normalfont Prof. Dr. Dr. hc. mult. Kerstin von Kienfeld}{Prof. Dr. Bernd Blume} % Erst- und Zweitgutachter der Arbeit
%\mitverteidigung % entfernen, falls keine Verteidigung erfolgt
\makeTitel

% Hier folgt die eigentliche Arbeit (bei doppelseitigem Druck auf einem neuen Blatt):
\setstretch{1,5}
\tableofcontents
\thispagestyle{empty}
\newpage

Insert Text here!

% Erzeugen der Selbständigkeitserklärung auf einem neuen Blatt:
%\selbstaendigkeitserklaerung{\today}

\end{document}
